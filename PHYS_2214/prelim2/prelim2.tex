\documentclass{hw}
\title{PHYS 2214 -- Prelim 2}
\author{community}

\usepackage{code}
\usepackage{pgf}
\usepackage{tikz}

\numberwithin{equation}{section}

%%%%%%%%%%%%%%%%%%%%%%%%%%%%%%%%%%%%%%%%%%%%%%%%%%%%%%%%%%%%%%%%%%%%%%%%%%%%%%%%
\begin{document}
\maketitle

\tableofcontents
\newpage{}

%%%%%%%%%%%%%%%%%%%%%%%%%%%%%%%%%%%%%%%%%%%%%%%%%%%%%%%%%%%%%%%%%%%%%%%%%%%%%%%%
\section{Dispersion}
Recall that in our discussion of ideal waves, we assumed a constant wave
velocity for a given medium that was invariant of the wave's shape or speed.
This simplifying assumption does not hold in real life. Instead, wave velocity
is a function of a wave's frequency, $\omega$, and wave vector, $k$.
\begin{equation}
\cbox{%
  v = \frac{\omega}{k}
}
\end{equation}

More precisely, we define the \emph{phase velocity}, $v_p$, and \emph{group
velocity}, $v_g$.
\begin{gather}
  \cbox{v_p = \frac{\omega}{k}} \\
  \cbox{v_g = \left. \frac{d\omega}{dk} \right|_{\omega, k}}
\end{gather}

If we consider a graph of $\omega$ and $k$ parametrized on $v$, we can form a
graphical description of the phase and group velocity. The phase velocity of a
pair $(\omega, k)$ is the slope of the line segment connecting the origin to
the point $(\omega, k)$. The group velocity of a pair $(\omega, k)$ is the
slope of the line tangent to the curve at $(\omega, k)$.

%%%%%%%%%%%%%%%%%%%%%%%%%%%%%%%%%%%%%%%%%%%%%%%%%%%%%%%%%%%%%%%%%%%%%%%%%%%%%%%%
\section{Beats}
Before we begin our analysis of beats, recall a trigonometric identity.
\begin{equation}\label{eq:beats-trig}
\cbox[blue]{%
  \cos(a) + \cos(b) = 2 \left[ 
    \cos\group{\frac{a+b}{2}} \cos\group{\frac{a-b}{2}} 
  \right]
}
\end{equation}

Now, consider two waves propagating in the same medium in the same direction.
\begin{align*}
  y_1 &= A \cos(k_1 x - \omega_1 t) \\
  y_2 &= A \cos(k_2 x - \omega_2 t)
\end{align*}

The superposition of the two waves is itself a wave.
\begin{align}
  y &= y_1 + y_2 \\
    &= \cbox{%
       2A \left[
        \cos\group{%
          \frac{\Delta k}{2}x - \frac{\Delta \omega}{2}t
        } 
        \cos\group{%
          \bar{k}x - \bar{\omega}t
        } 
      \right]
      }
\end{align}
\begin{equation}
  \Delta k = k_2 - k_1                         \quad 
  \Delta \omega = \omega_2 - \omega_1          \quad
  \bar{k} = \frac{k_1 + k_2}{2}                \quad
  \bar{\omega} = \frac{\omega_1 + \omega_2}{2} 
\end{equation}

If $\Delta k \ll k_1, k_2$, then the first sinusoid oscillates much slower than
the second and we get \emph{beats}, as shown in \figref{beats}. Notice that the
enveloping sinusoid has frequency $\frac{\Delta \omega}{2}$. However, the beat
frequency is half of the envelope's frequency. This is true because the beat is
maximally oscillating at both the peaks and valleys of the envelope.
\begin{equation}
  \omega_{\text{beat}} = \Delta \omega
\end{equation}

The beats are produced by the constructive and destructive interference, as
shown in \figref{beats-interference}.

\begin{figure}[ht]
  \centering
  \resizebox{\textwidth}{!}{\input{beats.pgf}}
  \caption{Example beats produced by \listref{beats}.}
  \label{fig:beats}
\end{figure}

\begin{figure}[ht]
  \centering
  \resizebox{\textwidth}{!}{\input{beats-interference.pgf}}
  \caption{Example beat interference produced by \listref{beats}.}
  \label{fig:beats-interference}
\end{figure}

%%%%%%%%%%%%%%%%%%%%%%%%%%%%%%%%%%%%%%%%%%%%%%%%%%%%%%%%%%%%%%%%%%%%%%%%%%%%%%%%
\section{Doppler Effect}
\subsection{Mechanical Waves}
Consider a source $s$ and observer $o$. $s$ travels with velocity $v_s$. $o$
travels with velocity $v_o$. $s$ emits a wave with frequency $f_s$ and velocity
$v$. From this, we can determine the wave frequency perceived by $o$, denoted
$f_o$.
\begin{equation}\label{eq:doppler}
\cbox{%
  f_0 = \frac{v + v_o}{v + v_s} f_s
}
\end{equation}

In equation~\eqref{eq:doppler}, $v_o$ and $v_s$ are positive when in the
direction of observer to source. However, there is an intuitive alternative to
memorizing this sign convention. Remember that $v_o$ appears in the numerator,
and remember that $v_s$ appears in the denominator. When thinking about the
velocity of a source or observer, ask yourself what effect their motion would
have on the observed frequency. Should the motion increase or decrease the
observed frequency? Based on this intuition, the sign of the velocity follows
naturally.

For example, consider a source moving away from the observer. If a source is
moving away, then the wave he emits will appear to have a lower frequency than
$f_s$. Thus, the observed frequency will decrease. Because $v_s$ appears in the
denominator of Equation~\eqref{eq:doppler}, $v_s$ must be positive to decrease
the value of $f_o$.

A few notes.
\begin{itemize}
  \item $v_o$ and $v_s$ are measured with respect to the medium in which the
    wave propagates.
  \item $v$ is always positive.
\end{itemize}

\subsection{Electromagnetic Waves}
TODO: There's some controversy on sign convention and a possible typo in the
lecture notes.

%%%%%%%%%%%%%%%%%%%%%%%%%%%%%%%%%%%%%%%%%%%%%%%%%%%%%%%%%%%%%%%%%%%%%%%%%%%%%%%%
\section{Energy in Mechanical Waves}
\subsection{Kinetic Energy}
Denote kinetic energy $U_k$ and kinetic energy density $u_k$.
\begin{gather}
  U_k = \frac{1}{2} m v^2 
      = \frac{1}{2} \mu \Delta x \group{\frac{\partial y}{\partial t}}^2 \\
  \cbox{%
  u_k = \frac{1}{2} \mu \group{\frac{\partial y}{\partial t}}^2
  }
\end{gather}

\subsection{Potential Energy}
Denote potential energy $U_p$ and potential energy density $u_p$.
\begin{gather}
  U_p = \frac{1}{2} \tau \Delta x \group{\frac{\partial y}{\partial x}}^2 \\
  \cbox{%
  u_p = \frac{1}{2} \tau \group{\frac{\partial y}{\partial x}}^2
  }
\end{gather}

\subsection{Total Energy}
For an arbitrary wave, 
\begin{equation}\label{eq:total-energy}
\cbox{%
  u = u_k + u_p =
      \frac{1}{2} \tau \group{\frac{\partial y}{\partial x}}^2
    + \frac{1}{2} \mu \group{\frac{\partial y}{\partial t}}^2
}
\end{equation}

For transverse travelling waves, we can use the pulse equation to simplify
Equation~\ref{eq:total-energy}.
\begin{equation}
\cbox{%
  u = \tau \group{\frac{\partial y}{\partial x}}^2
    = \mu \group{\frac{\partial y}{\partial t}}^2
}
\end{equation}

%%%%%%%%%%%%%%%%%%%%%%%%%%%%%%%%%%%%%%%%%%%%%%%%%%%%%%%%%%%%%%%%%%%%%%%%%%%%%%%%
\section{Power in Mechanical Waves}
\subsection{Impulse Waves}
In transverse mechanical waves, all particle motion is in the $y$ direction. We
can use this fact and the relationship $P = Fv$ to find a formula for the power
of a transverse mechanical wave.
\begin{align}
  P &= Fv \\
    &= \group{-\tau \frac{\partial y}{\partial x}} 
       \group{\frac{\partial y}{\partial t}} \\
    &= \cbox{%
      -\tau \frac{\partial y}{\partial x} \frac{\partial y}{\partial t}
    }
\end{align}

For a mechanical transverse pulse, we can apply the pulse equation to derive an
alternate formula for power.
\begin{equation}\label{eq:power-alternate}
\cbox{%
  P = vu
}
\end{equation}
Here $u$ is the total energy density as in Equation~\eqref{eq:total-energy}.

\subsection{Standing Waves}
To analyze the power of a mechanical standing wave, we can apply the general
power equation, 
  $P = -\tau \frac{\partial y}{\partial x} \frac{\partial y}{\partial t}$.

\paragraph{Nodes} First, analyze the power at a node. At a node, the medium
does not vertically oscillate. Thus,
\begin{equation}
  \forall t.\, \frac{\partial y}{\partial t} = 0
\end{equation}
However, the slope of the transverse wave reaches its maximum at a node. Thus 
\begin{equation}
  \exists t.\, 
  \frac{\partial y}{\partial x} = \max \group{\frac{\partial y}{\partial x}}
\end{equation}
From this, we can deduce that kinetic energy density is always zero and potential
energy density reaches its maximum at a node. Moreover, we can deduce that
power is always zero.
\begin{align}
  \forall t.\, P &= -\tau \frac{\partial y}{\partial x} 
                          \frac{\partial y}{\partial t}                   \\
                 &= -\tau \group{\frac{\partial y}{\partial x}} \group{0} \\
                 &= 0
\end{align}

\paragraph{Antinodes}
At an antinode, the slope of a standing wave is zero for all time, and the
vertical velocity of the wave realizes its maximum at the antinode. Using the
same logic as before, we can deduce three equations similar to that for a node.
\begin{gather}
  \forall t.\, 
    \frac{\partial y}{\partial x} = 0 \\
  \exists t.\, 
    \frac{\partial y}{\partial t} = \max \group{\frac{\partial y}{\partial t}} \\
  \forall t.\, 
    P = 0
\end{gather}

Thus, the power is always zero at nodes and antinodes. We can now claim that
there must not exist any net power transfer along a standing wave. For if there
were power transferred along a standing wave, then the power at nodes and
antinodes would have to be nonzero at some instant. This is a contradiction.

%%%%%%%%%%%%%%%%%%%%%%%%%%%%%%%%%%%%%%%%%%%%%%%%%%%%%%%%%%%%%%%%%%%%%%%%%%%%%%%%
\section{Listings}
\inputpython[%
  label=list:beats,
  caption=Beats
]{beats.py}

\end{document}
