\documentclass{hw}
\title{PHYS 2214 -- Final}
\author{community}

\usepackage{code}
\usepackage{pgf}
\usepackage{tikz}

\numberwithin{equation}{section}

%%%%%%%%%%%%%%%%%%%%%%%%%%%%%%%%%%%%%%%%%%%%%%%%%%%%%%%%%%%%%%%%%%%%%%%%%%%%%%%%
\begin{document}
\maketitle

\tableofcontents
\newpage{}

\section{Geometric Optics}

\section{Power and Momentum in EM Waves}
\subsection{Power and the Poynting Vector}
We have already explored the power of mechanical waves and found that it was
the product of the energy density and wave speed. We did not explore the
momentum of such waves but we can imagine that mechanical waves also have
momentum. This can be shown by running the simple experiment of setting a ball
in motion at the end of an oscillating string.

Similarly, in EM we can talk about energy and momentum. We first start with the
notion of the volumetric energy density of EM waves (Electric Field and Magnetic
Field Components):
\begin{equation}
\cbox{%
  u(t)  = \frac{1}{2}\epsilon_0 E(t)^2+\frac{1}{2\mu_0}B(t)^2
}
\end{equation}

$\epsilon_0$ and $\mu_0$ are properties of Electric Field and Magnetic fields
in a vacuum. More general constants are $\epsilon$ and $\mu$

Since we know that 
\begin{equation}
\cbox{%
  B(t) = \frac{E(t)}{c} = \sqrt{\epsilon_0 \mu_0} E(t)
}
\end{equation}
We can say that:
\begin{align}
  u(t) &= \frac{1}{2} \epsilon_0 E(t)^2 + \frac{1}{2 \mu_0} B(t)^2      \\
  u(t) &= \frac{1}{2} \epsilon_0 E(t)^2 +
          \frac{1}{2 \mu_0} \epsilon_0 \mu_0 E(t)^2                     \\
  u(t) &= \frac{1}{2} \epsilon_0 E(t)^2 + \frac{1}{2} \epsilon_0 E(t)^2 \\
  u(t) &= \cbox{\epsilon_0 E(t)^2}
\end{align}

Note that $E$ and $B$ are magnitudes of the Electric Field and Magnetic Field
respectively. Since these magnitudes are a function of position and time, the
energy density $u$ is also in general a function of position and time.

In EM it is often useful to talk about the \textit{erngy transfered per unit
time per unit cross-sectional area} or \textit{power per unit area}. The area
is perpendicular to the direction of the EM wave's propagation (velocity). We
now get that:

\begin{align}
  S &= \frac{\text{Power}}{{Area}}             \\
    &= \frac{dU}{dt} \cdot \frac{1}{A}         \\
    &= u(t) c                                  \\
    &= \epsilon_0 cE^2                         \\
    &= \epsilon_0 c E Bc                       \\
    &= \frac{\epsilon_0}{\epsilon_0 \mu_0} E B \\
    &= \cbox{\frac{\left|E\right| \left|B\right|}{\mu_0}} \label{eq:PoyntingMag}
\end{align}

Here we see that $S$ is a scalar quantity. If we define a vector quantity, we
get that:

\begin{equation}
\cbox{%
  \vec{S}(t) = \frac{1}{\mu_0} \vec{E}(t) \times \vec{B}(t)
}
\end{equation}

The vector quantity $\vec{S}$ represents the \textbf{Poynting Vector}. We can
know show Equation~\ref{eq:PoyntingMag} by noting that $\vec{E}$ and $\vec{B}$
are perpendicular. Also notice that the Poynting vector is a function of time.

The \textbf{Intensity} of an EM wave is defined as the average power per unit
area. In addition, by using the fact that $\vec{E}$ and $\vec{B}$ oscillate in
position and time, we can show that: 
\begin{equation}
\cbox{%
  Intensity = S_{av} = \frac{E_{max} B_{max}}{2\mu_0} = \frac{E_{max}^2}{2\mu_0
  c} = \frac{c\epsilon_0 E_{max}^2}{2}
}
\end{equation}
Here, the factor of $\frac{1}{2}$ comes from taking the average value of $E$
and $B$, both of which are sinusoidal.

\subsection{Momentum}
Electromagnetic waves have momentum, denoted $p$. Here, we consider the
momentum flow per unit area of an electromagnetic wave.
\begin{equation}\label{eq:radiation-pressure}
  \frac{\text{momentum flow}}{\text{area}} 
    = \frac{dp}{dt} \cdot \frac{1}{\text{area}}
\end{equation}

The units of Equation~\ref{eq:radiation-pressure} are $\frac{N}{m^2}$, or
pressure. Equation~\ref{eq:radiation-pressure} describes the \emph{radiation
pressure} of a wave: the pressure it exerts on a surface with which it
collides.

If the electromagnetic wave is completely absorbed by the surface it hits, then
\begin{equation}
  \cbox{\text{radiation pressure} = \frac{S}{c}}
\end{equation}
If the electromagnetic wave is reflected, then 
\begin{equation}
  \cbox{\text{radiation pressure} = \frac{2S}{c}}
\end{equation}


%%%%%%%%%%%%%%%%%%%%%%%%%%%%%%%%%%%%%%%%%%%%%%%%%%%%%%%%%%%%%%%%%%%%%%%%%%%%%%%%
\end{document}
