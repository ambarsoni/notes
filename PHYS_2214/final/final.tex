\documentclass{hw}
\title{PHYS 2214 -- Final}
\author{community}

\usepackage{code}
\usepackage{pgf}
\usepackage{tikz}

\numberwithin{equation}{section}

%%%%%%%%%%%%%%%%%%%%%%%%%%%%%%%%%%%%%%%%%%%%%%%%%%%%%%%%%%%%%%%%%%%%%%%%%%%%%%%%
\begin{document}
\maketitle

\tableofcontents
\newpage{}

\section{Geometric Optics}

\section{Power and Momentum in EM Waves}
We have already explored the power of mechanical waves and found that it was
the product of the energy density and wave speed. We did not explore the
momentum of such waves but we can imagine that mechanical waves also have
momentum. This can be shown by running the simple experiment of setting a ball
in motion at the end of an oscillating string.\\

Similarly, in EM we can talk about energy and momentum. We first start with the
notion of energy of EM waves(Electric Field and Magnetic Field Components):
\begin{equation}
\cbox{%
  u(t)  = \frac{1}{2}\epsilon_0 E(t)^2+\frac{1}{2\mu_0}B(t)^2
}
\end{equation}

$\epsilon_0$ and $\mu_0$ are propreties of Electric Field and Magnetic fields
in a vaccum. More general constants are $\epsilon$ and $\mu$

Since we know that 
\begin{equation}
\cbox{%
  B = \frac{E}{c} = \sqrt{\epsilon_0 \mu_0} E
}
\end{equation}

We can say that:

\begin{equation}
\cbox{%
  u(t) = \frac{1}{2}\epsilon_0 E^2
}
\end{equation}

Note that $E$ and $B$ are magnitudes of the Electric Field and Magnetic Field
respectively. Since these magnitudes are a function of position and time, the
energy density $u$ is also in general a function of position and time.\\ 

In EM it is often useful to talk about the \textit{erngy transfered per unit
time per unit cross-sectional area} or \textit{power per unit area}. The area
is perpendicular to the direction of the EM wave's propagation (velocity). We
now get that:

\begin{equation}
\cbox{%
  \label{eq::PoyntigMag}
  S = \frac{dU}{Adt} = \epsilon_0 cE^2 = \frac{\left|E\right| \left|B\right|}{\mu_0}
}
\end{equation}

Here we see that $S$ is a scalar quanitity. If we define a vector quanitity, we
get that:

\begin{equation}
\cbox{%
  \vec{S} = \frac{1}{\mu_0} \vec{E} \times \vec{B}
}
\end{equation}

The vector quantity $\vec{S}$ represents the \textbf{Poyntig Vector}. We can
know show equation \ref{PoyntigMag} by noting that $\vec{E}$ and $\vec{B}$ are
perpendicular. \\

The \textbf{Intensity} of an EM wave is defined as the average power per unit
area. In addition, by using the fact that $\vec{E}$ and $\vec{B}$ oscillate in
position and time, we can show that: 
\begin{equation}
\cbox{%
  Intensity = S_{av} = \frac{E_{max} B_{max}}{2\mu_0} = \frac{E_{max}^2}{2\mu_0
  c} = \frac{c\epsilon_0 E_{max}^2}{2}
}
\end{equation}



%%%%%%%%%%%%%%%%%%%%%%%%%%%%%%%%%%%%%%%%%%%%%%%%%%%%%%%%%%%%%%%%%%%%%%%%%%%%%%%%
\end{document}
